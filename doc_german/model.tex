\section{Datenmodell}
\label{datenmodell}

Die Datenstruktur kann in einem einfach gehaltenen \textit{Entity-Relationship}-Modell dargestellt werden. Im Wesentlichen besteht die Basis-Entität \texttt{Reference} aus grundlegenden Container-Attributen, die an essentiellen Stellen mit Informationen angereichert wurden. Die Datenstruktur wird in diesem Kapitel kurz erläutert.

Entity Relation Model

Den übergeordneten Datencontainer bildet die \texttt{Reference}-Klasse, sie hält zwei Attribute, \texttt{OSM::Location} und eine Liste von \texttt{Discovery}-Instanzen vor. Eine Georeferenz (also eine Instanz der hier beschriebenen  \texttt{Reference}-Klasse) liefert einen Verweis auf einen geografischen, real existierenden Ort auf der Welt und eine Liste von Hinweisen, wie und wo diese Referenz durch etwaige Suchalgorithmen gefunden und in den entsprechend gefundenen Dokumenten (beispielsweise auf Webseiten oder in semantischen Quellen wie DBpedia "referenziert" wurde.

Die \texttt{OSM::Location}-Klasse stellt eine sehr starke Abstraktion einer OpenStreetMaps \texttt{Node}\cite{OSMnode} dar. Im OpenStreetMaps-Datenmodell repräsentieren Nodes einen distinkten Punkt, der innerhalb des Systems einzigartig ist und mit Positionsdatenversehen ist. In unserem Datenmodell sind die konkreten Geokoordinaten nicht von essentieller Wichtigkeit (weil der Fokus auf Flächen, Regionen, Städten liegt, wichtig ist uns die Bennenbarkeit), daher persistieren wir in der \texttt{OSM::Location}-Klasse lediglich die \texttt{node\_id}, einen menschenlesbaren Namen und eine Repräsentation des Administration-Level-Konzepts von OpenStreetMaps\cite{OSMadminlevel}. Mit dem Administration-Level kann bestimmt werden, ob es sich bei der angegebenen Georeferenz etwa um einen Kontinent, einen Staat, ein Bundesland oder eine Stadt handelt.

Weiterhin enthält die \texttt{Reference}-Klasse wie erwähnt eine Liste von \texttt{Discovery}-Instanzen. In einer \texttt{Discovery}-Instanz wird festgehalten, wo (\texttt{Source}-Klasse) und in welchem zeitlichen Rahmen (\texttt{Time}-Klasse) diese Georeferenz an dem durch die \texttt{OSM::Location}-Klasse konkret festgelegten Ort referenziert wird. Weiterhin gibt es eine \texttt{Discovery::Locator}-Klasse, die die Rückverfolgbarkeit der Quelle sicherstellt. Die einzelnen Discovery-Komponenten werden im Folgenden noch genauer geschrieben.

Die \texttt{Discovery::Source}-Klasse ist ein Container für zwei Attribute: Zum einen hält ein \texttt{identifier} eine Referenz auf eine konkrete Datenquelle vor, beispielsweise \textit{dbp} für DBpedia. Zum anderen hält der \texttt{trustworthiness}-Wert eine Einschätzung für die Zuverlässigkeit der Quelle vor. Beide Attribute ergeben sich aus dem Anwendungsfall und müssen pro Durchgang konfigurierbar gehalten sein. Der \texttt{trustworthiness}-Wert enthält Werte einer Likert-Skala (von "Sehr vertrauenswürdig" bis "Nahezu nicht vertrauenswürdig"), intern können die Werte dieser Skala auf Ganzzahlen (oder Enumeration-Werte) abgebildet werden, allerdings verbietet sich das Rechnen auf Likert-Zahlen.

Die \texttt{Discovery::Time}-Klasse ist ein Container für die beiden Attribute \texttt{representation} und \texttt{estimated\_precision}. Der erste Wert gibt eine konkrete, wörtliche Fundstelle einer Zeitangabe wieder (also beispielsweise "1990ies", "1997" oder "1990-1996"). Der zweite Wert gibt auf einer Likert-Skala (von "sehr präzise" bis "sehr unpräzise") eine Einschätzung über die Genauigkeit der Zeitangabe - in Hinblick auf den Anwendungsfall - an. So wäre beispielsweise im Rahmen einer Epochenbeschreibung eine Jahresangabe "sehr präzise" und eine Angabe eines Jahrtausends "sehr unpräzise". Selbige Einschränkungen wie oben beschrieben in Hinblick auf die Likert-Skalen gelten auch hier. Die Einschätzung der Präzision muss ein Algorithmus liefern.

Der \texttt{Discovery::Locator} ist eine Klasse mit Elementen, die auf den genauen Fundort einer Georeferenz in einem Web-Dokument (oder ähnlichem) verweist. Ziel hierbei ist, im späteren Verlauf den Ursprung der Fundstelle maschinell oder durch den Menschen nachverfolgbar (oder belegbar, beispielsweise in einem journalistischen Anwendungsszenario) zu machen. Konkret beinhaltet die \texttt{Discovery::Locator}-Klasse die Felder \texttt{url} um mit dort eine Resource im Internet unabhängig von einem konkreten Schema referenzieren zu können, sowie ein \texttt{byte\_offset}, um die konkrete Stelle im Dokument, das durch die URL referenziert ist, anzuzeigen, als auch einen \texttt{timestamp}, um den Zeitpunkt des Fundes festzuhalten.
  
Es existiert weiterhin ein optionales Feld \texttt{proof}, das die Fund-Webresource base64-enkodiert in Kopie vorhält.

Alle oben genannten Klassen enthalten beliebig weitere Attribute, die vom Namen her unterschiedlich zu den bereits vorgestellten Attributen sein müssen. Hier hat der Entwickler die Möglichkeit anwendungsspezifische Daten in die Datenstruktur einzubetten um beispielsweise Geokodierungsschritte oder Umwandlungen einzusparen oder zusätzliche Meta-Informationen mitzuliefern.

% TODO Add citations for OSM
