\section{Technologien}
\label{technologies}
Es liegt nahe, für die beschriebene Datenstruktur Schnittstellen bereitzustellen und bestehende Technologien zu verwenden, die die weiter oben genannten Ziele möglichst effizient und entwicklerfreundlich unterstützen. Hier soll kurz beschrieben werden, welche technischen Herangehensweisen in der Verwendung der genannten Datenstruktur besonders sinnvoll erscheinen.

\subsection{Anbindung an bestehende Geodatenbanken}
Es ist häufig von geografischen Referenzen die Rede, daher ist es naheliegend, bestehende Geodatenbanken wie OpenStreetMaps einzubinden, um Geo-Locations referenzieren zu können. Die Datenstruktur bedient sich, wie oben bereits erwähnt, einiger aus OpenStreetMaps entlehnter Konzepte, die direkt auf das sehr umfangreiche und freie Kartenprojekt abgebildet werden können. OpenStreetMaps selbst basiert zum überwiegenden Teil auf XML-Formaten, was die Verzahnung mit weiteren Webprojekten noch weiter vereinfachen kann.

\subsection{Verwendung stringorienterter Programmiersprachen}
Die Referenzimplementierung der Datenstruktur liegt in Ruby vor\cite{georeferencegem}. Ruby ist eine Programmiersprache, die neben ihrem Duck-Typing-Konzept auch für effiziente Ansätze bei Zeichenkettenverarbeitung bekannt ist. Beide Konzepte werden dann nützlich, wenn unterschiedlichste textbasierte Datenquellen eingelesen, ausgewertet und zu einer Datenkomponente zusammengefasst werden sollen. Die Offenheit der Implementierung kann vorteilhaft sein, wenn Attribute oder standardisierte Methoden nachgefügt werden sollen. Dies ist beispielsweise auch dann der Fall, wenn zukünftig \textit{Reference}-Instanzen über Anwendungsprogrammierungsschnittstellen (\textit{API}s) weitergegeben werden sollen. Obwohl die Referenzimplementierung in Ruby gehalten ist, behandelt dieses Dokument doch vorrangig eine vergleichsweise einfache Datenstruktur, die so oder ähnlich auch ein einer beliebigen anderen Programmiersprache implementiert werden könnte. So wäre vorstellbar Python als ebenso String-optimierte Sprache zu verwenden, oder Java - eine Sprache, in der ein Klassenmodell restriktiver als hier gefordert einfacher durchgesetzt werden könnte.

\subsection{JSON-Schnittstelle}
Im Regelfall sind auf Datenstrukturen keine Methoden oder Funktionen definiert. Dennoch ist es sinnvoll Empfehlungen für die Erreichung der genannten Ziele auszusprechen: Als einzig derzeitig wichtige standardisierte Methode auf dem Datentyp ist \texttt{to\_json} zu nennen, die die Felder aller Strukturen und Unterstrukturen in Javascript Object Notation\cite{jsonspec} zurück gibt. Dies ist sehr sinnvoll, für den Fall, dass Instanzen dieser Datenstrukturen typischerweise über \textit{Web-API}s geliefert werden sollen. Für die Heuristiken - als theoretisches Konstrukt, die diese Datenstrukturen bereitstellen - ist \texttt{to\_json} aber gewiss ohne Relevanz, findet sich aber beispielsweise in der Referenzimplementierung.
