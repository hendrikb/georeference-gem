\documentclass[twocolumn,10pt]{asme2ej}
\usepackage{epsfig} %% for loading postscript figures
\title{} 
%%% first author
\author{Hendrik Bergunde
    \affiliation{
	Masterstudent, Freie Universität Berlin\\
    Email: hendrik.bergunde@fu-berlin.de
    }	
}
%%% second author
\author{Frank Zechert
    \affiliation{
	Masterstudent, Freie Universität Berlin\\
    Email: frank.zechert@fu-berlin.de
    }	
}
\begin{document}

\maketitle    

\begin{abstract}
{\it Abstract}
\end{abstract}

Motivation \& Ziele
  Georeferenzen aus verschiedenen Systemumgebungen abbildbar, vergleichbar, zusammenführbar machen
  Dazu Vertrauen und Genauigkeit (Qualitätskriterien?) ausdrücken
  

Entity Relation Model

  Beschreibung der einzelnen Entitäten
    Referenz (nur "Rahmen")
    OSM-Location (node\_id und OSM Layer Verlinkung)
    Discovery (Rahmen für S und T)
    Discovery::Locator (URL, byteindex)
        "url": http://wikipedia.org/JebBush
        "byte\_offset": 12345 % optional
        "timestamp": ...
        "proof": { payload\_base64: "...", mimetype: "text/html" }
    Discovery::Source
    Discovery::Time

    Begründung Existenz anwendungsspezifische Felder überall (optionals) als K/V in das jeweilige Objekt reingehängt

  Festlegen der Skalen (TW)

"Probleme"
   Warum zB Likertskalen? 
   Rückverfolgbarkeit der Treffer (Quellenangabe?)

"Assoziierte" Technologien 
  OSM (-> schnell bei XML) % http://wiki.openstreetmap.org/wiki/Node
  JSON

\bibliographystyle{asmems4}
\bibliography{paper}
\end{document}
