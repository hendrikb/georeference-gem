\documentclass[twocolumn,10pt]{asme2ej}
\usepackage{epsfig} %% for loading postscript figures
\usepackage[utf8]{inputenc}
\usepackage{float}
\usepackage{tikz}
\usepackage{tikz-uml}
\usepackage[backend=biber]{biblatex}


\title{Ein Datenmodell für die Geolokalisierung von Akteuren} 
%%% first author
\author{Hendrik Bergunde
    \affiliation{
	Masterstudent, Freie Universität Berlin\\
    Email: hendrik.bergunde@fu-berlin.de
    }	
}
%%% second author
\author{Frank Zechert
    \affiliation{
	Masterstudent, Freie Universität Berlin\\
    Email: frank.zechert@fu-berlin.de
    }	
}
\begin{document}

\maketitle    

\begin{abstract}
{\it Akteure, zum Beispiel Firmen oder Personen, sind im Zeitraum ihres Lebens oder ihrer Existenz an unterschiedlichen Orten aktiv. Möchte man diese Akteure mit Hilfe von verschiedenen Techniken und Programmen lokalisieren, benötigt man ein einheitliches Datenmodell, in dem die Ergebnisse repräsentiert werden. Nur dann ist es möglich die verschiedenen Ergebnisse zu vereinigen und gemeinsam auszuwerten. In diesem Paper stellen wir ein Datenmodell für diesen Zweck vor, welches auf der einen Seite allgemein genug ist, um unabhängig von der verwendeten Technik zu sein, und auf der anderen Seite die Ergebnisse trotzdem mit wichtigen Qualitätsmerkmalen ausstattet.}
\end{abstract}

\section{Einführung}
\label{introduction}
Es existiert ein Anwendungsfall, in dem ein Algorithmus verschiedenste Quellen im Internet nach benannten Orten (real existierende geografische Entitäten: Städte, Länder, Kontinente, etc.) durchsuchen soll. Diese Orte werden anschließend zueinander in Verbindung gebracht.

Ziel ist es natürlich, möglichst viele Quellen anbinden zu können. So sollen beispielsweise Texte in der Wikipedia oder im freien Internet (z.B. ausgehend von Suchmaschinen-Ergebnisseiten) nach benannten Orten durchsucht werden. Die Voraussetzung dafür ist aber, dass dieser \textit{Scraping}-Algorithmus verschiedenste Quellen anbinden kann.

Um diese Fundstellen aus unterschiedlichen Quellen vorhalten zu können, muss es eine Datenstruktur geben, die gewährleistet, dass - egal aus welche Quelle eine dieser Georeferenzen stammt - die Ergebnisse gleichartig gespeichert, weitergegeben und insbesondere miteinander verglichen werden können.

Diese Heuristik soll hier von verschiedensten Blickwinkeln aus betrachtet und erläutert werden.

\section{Motivation und Ziele}
Motivation \& Ziele
  Georeferenzen aus verschiedenen Systemumgebungen abbildbar, vergleichbar, zusammenführbar machen
  Dazu Vertrauen und Genauigkeit (Qualitätskriterien?) ausdrücken
\section{Technische Herausforderungen}
Die effiziente Geolokalisierung von Akteuren in einem sozialen Graph soll mit Hilfe von Computerprogrammen erfolgen. Sofern für diese Aufgabe keine besondere spezialisierte Informationsquelle zur Verfügung steht, werden hierfür im Allgemeinen Informationen aus dem Internet verwendet. Abhängig vom Zutrauen zur Korrektheit der Informationsquelle und dem Extraktionsverfahren mit dem die Informationen ermittelt werden, ändert sich die Qualität des Ergebnisses. Für eine problemlose Weiterverwendung der Daten aus einem Geolokalisierungsprozess ist es unabdingbar, dass die Qualität der Ergebnisse einschätzbar ist.

\subsection{Güte einer Quelle}
Die Bewertung der Güte einer Informationsquelle ist zumindest teilweise eine subjektive Entscheidung, die sich im besten Fall auf möglichst viele Indizien stützt. Da unser Datenmodell außerdem für mehrere Implementierungen einer Geolokalisierung übergreifend gültig sein soll, ist es nicht sinnvoll die Güte einer Informationsquelle in einer zu granularen Einheit abzubilden. Gibt man die Güte einer Quelle zum Beispiel mit einer Fließkommazahl zwischen 0 (nicht vertrauenswürdig) bis 1 (voll vertrauenswürdig) an, können die Indikatoren von verschiedenen Quellen sehr nahe beieinander liegen (z.B. $0.8$ und $0.83$). Auf Grund der subjektiven Anteile in der Bewertung und unterschiedlicher Implementierungen der Informationsextraktion in unterschiedlichen Programmen sagt ein so geringer Unterschied nichts darüber aus, welche Quelle eine bessere Güte hat. Wir haben uns daher stattdessen dazu entschieden, die Güte einer Quelle in relativ groben Einheiten zu unterscheiden, wie zum Beispiel ``sehr vertrauenswürdig'', ``vertrauenswürdig'', ``nicht vertrauenswürdig''. Da eine solch grobe Einteilung ein Rechnen\label{calc_likert} mit den Güteindikatoren verhindert und die Abstände zwischen den Einheiten unbestimmt groß sein können, bieten sich hierfür Likert-Skalen an. Eine genaue Festlegung findet sich im Abschnitt \ref{datenmodell} ``Datenmodell'' auf Seite \pageref{datenmodell}.

\subsection{Präzision einer Datumsangabe}
Neben der Korrektheit einer Information aus einer Quelle spielt auch der Zeitpunkt eine Rolle, auf den sich eine Information aus einer Quelle bezieht. Akteure, deren aufenthaltsorte man mittels Geolokalisierung finden möchte, bewegen sich im Laufe der Zeit zwischen verschiedenen Orten. Die Information, dass sich ein Akteur in New York befindet, ist erst dann hinreichend aussagekräftig, wenn erkennbar ist, in welchem Zeitraum dies der Fall war. Im günstigsten Falle ist eine solche Information mit einem konkreten Datum (Zeitpunkt) oder mit eindeutigen Anfangs- und Enddaten (Zeitintervall) annotiert. Ist dem nicht so, kann die Aussagekraft dieser Information stark nachlassen. Ohne jegliche Angabe eines Zeitpunktes für einen Auffenthaltsort einer Person, lässt sich nur die Aussage treffen, dass eine Person zwischen ihrem Geburts- und Todesdatum zu einem beliebigen Zeitpunkt an einem Ort war. Ähnlich zu den Likert-Skalen für die Qualität der Informationsquelle ist es auch hier sinnvoll, eine etwas gröbere Abstufung für die Genauigkeit des Zeitpunktes anzugeben. In diesem Fall beginnend bei der genauesten Angabe, nämlich ein konkreter ``Zeitstempel'', über etwas ungenauere Angaben wie ``Jahr`` oder ``Jahrzehnt`` bis hin zur ungenauesten Angabe ``Jahrhundert``. Zeitangaben, die noch unpräziser sind, werden hier nicht weiter beachtet. An dieser Stelle ist wiederum ein Hinweis auf den Anwendungsfall sinnvoll, je nach Zweck der Implementierung kann es entegegen der ersten Intuition sinnvoll sein, beispielsweise Zeitintervallen Vorrang vor konkreten Zeitpunkten zu geben. Hier sollte der Algorithmus, der die Datenstruktur befüllt und die Genauigkeit bewertet, konfigurierbar sein.

\subsection{Belegbarkeit der Informationen}
In aller Regel ist es von großem Interesse, herausfinden zu können von wo eine Information, die für die Geolokalisierung genutzt wurde, stammt. Zum Beispiel wenn man die Arbeit eines Geolokalisierungsprogrammes auf Korrektheit prüfen möchte, indem man die Informationen manuell zurückverfolgt und bewertet. Das führt bei Informationen, die aus dem schnellebigen Internet stammen zu zwei wesentlichen Problemen. \textbf{1}. Die Informationen aus dem Internet können sich schnell ändern oder gänzlich nur für kurze Zeit verfügbar sein. Es ist daher möglich, dass zum Zeitpunkt der Geolokalisierung eine andere Information in einer Quelle existiert hat, als zu dem Zeitpunkt, an dem man eine manuelle Überprüfung der Quelle vornehmen möchte. Aus diesem Grund sollte es möglich sein, den Beleg für eine Information (zum Beispiel Kopie einer Webseite), direkt in den Ergebnissen anzugeben. \textbf{2}. Das Internet hält größtenteils unstrukturierte Informationen bereit, die zum Beispiel mit Hilfe von Natural Language Processing erst maschinentauglich auswertbar gemacht werden müssen. Eine einzelne Quelle (``Webseite'') kann zudem mehrere Informationen enthalten. Man benötigt eine Möglichkeit, den Ort einer Quelle genauer anzugeben, als nur eine URL anzugeben. Auf Grund der Unstruktiertheit des Internets ist die einzige übergreifend mögliche Methode hierfür die Angabe eines Bytes, an dem (oder ab dem) eine Information gefunden wurde.

Die Angabe einer Kopie der Quelle sowie der Zeiger auf eine Stelle in dieser Quelle (Byte-Offset) ist ein eindeutiger Verweis auf die Informationsherkunft. Je nach Quelle kann eine solche Angabe jedoch sehr speicherintensiv werden, da die Kopie der Quelle sehr groß sein kann. Eine Quelle muss nicht unbedingt eine Textdatei sein (HTML, XML, ...), sondern kann jede beliebige Datei aus dem Internet sein. Dazu gehören auch große Binärdateien wie Bilder oder Videos, sofern ein Prozess aus diesen Dateien wichtige Informationen extrahieren kann. Aus diesem Grund soll eine solche Angabe in unserem Datenmodell optional und nur bei Bedarf aktivierbar sein.

\subsection{Einfache Repräsentation der Daten}
Um die Daten möglichst einfach zu repräsentieren und sie zwischen Prozessen austauschbar zu machen, empfehlen wir für dieses Modell JSON als Datenformat. JSON ist neben XML ein weit verbreitetes Datenformat in der Internetlandschaft, lässt sich aber einfacher und resourcensparender als XML durch andere Prozesse wieder einlesen. Das JSON Format unterstützt keine Binärdaten. Für die optionale Angabe einer Quelle in der Ausgabe ist es daher notwendig, die Quelle zu kodieren. Dafür nutzen wir in unserem Datenmodell die ``base64'' Kodierung. Dadurch kann die Größe einer Quellenkopie um etwa $33\%$ ansteigen \cite{ng2005study}. Ein weiterer Grund dafür, warum diese Angabe im Datenmodell optional ist. Eine weitere Übersicht über die verwendeten Technologien findet sich im Abschnitt \ref{technologies} ``Technologien''. Die Referenzimplementierung bietet bereit JSON-Schnittstellen.

\section{Assoziierte Technologien}
"Assoziierte" Technologien 
OSM (-> schnell bei XML) % http://wiki.openstreetmap.org/wiki/Node
JSON
\section{Datenmodell}
\label{datenmodell}

Die Datenstruktur kann in einem einfach gehaltenen \textit{UML-Klassendiagramm}\ref{fig:uml_diagram} dargestellt werden. Im Wesentlichen enthält die Basis-Entität \texttt{Reference}  grundlegendene Container-Attribute, die wiederum mit Informationen angereichert wurden. Die Datenstruktur wird in diesem Kapitel kurz erläutert.

\tikzumlset{fill class=red!20, fill template=violet!10, font=\tiny} 
\begin{figure}[H]
  \centering
  \begin{tikzpicture} 
    \begin{umlpackage}[x=0,y=0]{Geo} 
      \umlclass[x=0,y=9]{Reference}{ 
        + location : Geo::OSM::Location \\ + discoveries : Geo::Discovery::Base[] \\ + additional\_fields : String[][]
        }{} 
        \begin{umlpackage}[x=4,y=8.1]{OSM} 
          \umlclass[x=0,y=0]{Location}{}{
            + node\_id : int \\ + human\_readable\_name : String  \\ + administrative\_level : int \\ + additional\_fields : String[][]
          }
        \end{umlpackage}
        \begin{umlpackage}[x=0,y=0]{Discovery} 
          \umlclass[x=1.5,y=5.6]{Base}{ 
            + source : Source \\ + time : Time  \\ + locator : Locator \\ + additional\_fields : String[][]
            }{} 
          \umlclass[x=0,y=2]{Source}{ 
            + identifier : String \\ + trustworthiness : int  \\ + additional\_fields : String[][]
            }{} 
          \umlclass[x=3,y=2]{Time}{ 
            + representation : String \\ + estimated\_precision : int  \\ + additional\_fields : String[][]
            }{} 
          \umlclass[x=1.5,y=-0.45]{Locator}{ 
            + url : String \\ + byte\_offset : int  \\ + timestamp : DateTime \\ + proof : String \\ + additional\_fields : String[][]
            }{} 
        \end{umlpackage} 
    \end{umlpackage} 
    \umlunicompo[pos1=-0.1,arg1=location,mult1=1,mult2=1,pos2=0.95]{Reference}{Location} 
    \umluniaggreg[arg1=discoveries,mult1=1,mult2=*,pos2=0.8]{Reference}{Base} 

    \umlunicompo[arg1=source,pos1=0.4, mult2=1,pos2=0.8]{Base}{Source} 
    \umlunicompo[arg1=locator, pos1=0.185, mult2=1,pos2=0.95]{Base}{Locator} 
    \umlunicompo[arg1=time, pos1=0.2, mult2=1,pos2=0.8]{Base}{Time} 
  \end{tikzpicture}
\caption{UML-Klassendiagramm der Datenstruktur}
\end{figure}


\subsection{Reference-Containerklasse}
Den übergeordneten Datencontainer bildet die \texttt{Reference}-Klasse, sie hält zwei Attribute:

\begin{enumerate}
   \item Das \texttt{location}-Attribut beinhaltet eine Instanz der \texttt{OSM::Location}-Klasse. 
   \item Das \texttt{discoveries}-Attribut hält eine Liste von \texttt{Discovery}-Instanzen. 
\end{enumerate}

Eine Georeferenz, also eine Instanz der hier beschriebenen \texttt{Reference}-Klasse, liefert dementsprechend einen Verweis auf einen geografisch real existierenden Ort auf der Welt und eine Liste von Hinweisen, wie und wo diese Referenz durch etwaige Suchalgorithmen in Webdokumenten gefunden und entsprechend referenziert wurde.

\subsection{OSM::Location: OpenStreetMaps-basierte Ortsangabe}
Die \texttt{OSM::Location}-Instanz stellt eine stark vereinfachte Abstraktion einer OpenStreetMaps \texttt{Node}\cite{OSMnode} dar. Im OpenStreetMaps-Datenmodell repräsentieren Nodes einen distinkten Punkt, der innerhalb von OpenStreetMaps einzigartig ist und mit Positionsdaten versehen ist. In unserem Datenmodell sind die exakten Geokoordinaten nicht von essentieller Wichtigkeit (weil der Fokus eher auf Städten, Regionen oder Ländern liegt, wichtig ist uns die Bennenbarkeit eines Ortes, nicht dessen Lage), daher persistieren wir in der \texttt{OSM::Location}-Instanz lediglich \ldots

\begin{enumerate}
  \item die \texttt{node\_id}, sowie 
  \item einen menschenlesbaren Namen (Attribut \texttt{human\_readable\_name}) und 
  \item den \textit{Administration-Level}-Wert von OpenStreetMaps\cite{OSMadminlevel} (Attribut \texttt{administrative\_level}). 
\end{enumerate}

Mit Letzterem kann bestimmt werden, ob es sich bei der benannten Georeferenz etwa um einen Kontinent, einen Staat, ein Bundesland oder eine Stadt handelt.

\subsection{Inhaltliche Beschreibung einer Georeferenz: Discovery-Klasse}
Die \texttt{Reference}-Klasse enthält wie erwähnt eine Liste von \texttt{Discovery}-Instanzen. In einer \texttt{Discovery}-Instanz wird festgehalten, \textit{wo} (Attribut: \texttt{source}, Instanz einer \texttt{Discovery::Source}-Klasse) und in welchem \textit{zeitlichen Rahmen} (Attribut \texttt{time}, Instanz einer \texttt{Discovery::Time}-Klasse) diese Georeferenz an dem durch die \texttt{OSM::Location}-Klasse konkret festgelegten Ort referenziert wird. Weiterhin gibt es eine \texttt{Discovery::Locator}-Klasse, die die Rückverfolgbarkeit der Quelle sicherstellt. Die einzelnen Discovery-Komponenten werden im Folgenden noch genauer geschrieben.

\subsubsection{Informationen zur Quelle: Discovery::Source}
Die \texttt{Discovery::Source}-Klasse ist ein Container für zwei Attribute: Zum einen hält ein \texttt{identifier} eine Referenz auf eine konkrete Datenquelle vor, beispielsweise \textit{dbp} für DBpedia. Zum anderen hält der \texttt{trustworthiness}-Wert eine Einschätzung für die Zuverlässigkeit der Quelle vor. Beide Attribute ergeben sich aus dem Anwendungsfall und müssen pro Anwendung konfigurierbar gehalten sein. Der \texttt{trustworthiness}-Wert enthält Werte einer Likert-Skala (von ``Sehr vertrauenswürdig'' bis ``Nicht vertrauenswürdig''), intern könnten die Werte dieser Skala auf Ganzzahlen (oder Enumeration-Werte) abgebildet werden, allerdings verbietet sich das Rechnen auf Likert-Skalen aus den auf Seite \pageref{calc_likert} genannten Gründen.

\subsubsection{Zeitreferenz: Discovery::Time}
Die \texttt{Discovery::Time}-Klasse ist ein Container für die beiden Attribute \texttt{representation} und \texttt{estimated\_precision}. Der erste Wert gibt eine konkrete, wörtliche Beschreibung einer Zeitangabe wieder (also beispielsweise ``1990ies'', ``1997'' oder ``1990-1996''). Der zweite Wert gibt auf einer Likert-Skala (von ``sehr präzise'' bis ``sehr unpräzise'') eine Einschätzung über die Genauigkeit der Zeitangabe - in Hinblick auf den Anwendungsfall - an. So wäre beispielsweise im Rahmen einer Epochenbeschreibung eine Jahresangabe ``sehr präzise'' und eine Angabe eines Jahrtausends ``sehr unpräzise''. Selbige Einschränkungen wie oben beschrieben in Hinblick auf die Likert-Skalen gelten auch hier. Die Einschätzung der Präzision muss ein, die Datenstruktur befüllender, Algorithmus liefern. Hintergrund ist, dass die Feinjustierung dieser Einschätzung von Anwendungsfall zu Anwendungsfall unterschiedlich sein kann. So wird es Szenarien geben, in denen konkrete Tages-Angaben von Bedeutung sind, andererseits wird es Fälle geben, in denen Jahresangaben oder sogar Jahrzehnte ``präzise'' genug sind. Auch hier ist also die Konfigurierbarkeit essentiell.

\subsubsection{Spezifizierung der Fundstelle: Discovery::Locator}
Der \texttt{Discovery::Locator} ist eine Klasse mit Elementen, die auf den genauen Fundort einer Georeferenz in einem Web-Dokument (oder ähnlichem) verweist. Ziel hierbei ist, im späteren Verlauf den Ursprung der Fundstelle maschinell oder durch den Menschen nachverfolgbar (oder belegbar, beispielsweise in einem journalistischen Anwendungsszenario) zu machen. Konkret beinhaltet die \texttt{Discovery::Locator}-Klasse die Felder \ldots

\begin{enumerate}
  \item{\texttt{url} um mit dort eine Resource im Internet unabhängig von einem konkreten Schema referenzieren zu können, sowie}
  \item{ein \texttt{byte\_offset}, um die konkrete Stelle im Dokument, das durch die URL referenziert ist, anzuzeigen, als auch}
  \item{einen \texttt{timestamp}, um den Zeitpunkt des Fundes festzuhalten.}
\end{enumerate}
  
Es existiert weiterhin ein optionales Feld \texttt{proof}, das die Fund-Webresource base64-enkodiert in Kopie vorhalten kann.

\subsection{Anwendungsfallspezifische Zusatzattribute}

Alle oben genannten Klassen enthalten beliebig viele weitere Attribute, die vom Namen her unterschiedlich zu den bereits vorgestellten Attributen sein müssen. Hier hat der Entwickler die Möglichkeit anwendungsspezifische Daten in die Datenstruktur einzubetten um beispielsweise Geokodierungsschritte oder Umwandlungen einzusparen oder zusätzliche Meta-Informationen mitzuliefern.




\printbibliography
\end{document}
