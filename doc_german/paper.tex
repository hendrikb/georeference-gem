\documentclass[twocolumn,10pt]{asme2ej}
\usepackage{epsfig} %% for loading postscript figures
\usepackage[utf8]{inputenc}
\usepackage{float}
\usepackage{tikz}
\usepackage{tikz-uml}
\title{Ein Datenmodell für die Geolokalisierung von Akteuren} 
%%% first author
\author{Hendrik Bergunde
    \affiliation{
	Masterstudent, Freie Universität Berlin\\
    Email: hendrik.bergunde@fu-berlin.de
    }	
}
%%% second author
\author{Frank Zechert
    \affiliation{
	Masterstudent, Freie Universität Berlin\\
    Email: frank.zechert@fu-berlin.de
    }	
}
\begin{document}

\maketitle    

\begin{abstract}
{\it Akteure, zum Beispiel Firmen oder Personen, sind im Zeitraum ihres Lebens oder ihrer Existenz an unterschiedlichen Orten aktiv. Möchte man diese Akteure mit Hilfe von verschiedenen Techniken und Programmen lokalisieren, benötigt man ein einheitliches Datenmodell, in dem die Ergebnisse repräsentiert werden. Nur dann ist es möglich die verschiedenen Ergebnisse zu vereinigen und gemeinsam auszuwerten. In diesem Paper stellen wir ein Datenmodell für diesen Zweck vor, welches auf der einen Seite allgemein genug ist, um unabhängig von der verwendeten Technik zu sein, und auf der anderen Seite die Ergebnisse trotzdem mit wichtigen Qualitätsmerkmalen ausstattet.}
\end{abstract}

\section{Motivation und Ziele}

Die Geolokalisierung von Akteuren mit Hilfe von Daten aus dem Internet kann auf verschiedenste Weise maschinell oder händisch durchgeführt werden. Ziel dabei ist immer die extraktion von Informationen, die auf den Aufenthaltsort des Akteurs hinweise geben. Wenn verschiedene Systeme und Verfahren mit dieser Aufgabe betraut werden, dann benötigt man ein einheitliches Datenformat für die im Ergebnis gefundenen Georeferenzen. Denn möchte man auf einer möglichst vollständigen Datenmenge an Georeferenzen arbeiten, muss man die Ergebnisse von verschiedenen Methoden zusammenführen und miteinander vergleichen können.

Wie genau verschiedene Ergebnisse miteinander verglichen oder zusammengeführt werden, muss hierbei jedoch immer vom Anwendungsfall abhängen. Je nach Anwendungsfall kann sich das Interesse an unterschiedlichen Qualitätskriterien der Ergebnisse stark unterscheiden. Wir wollen daher eine Datenstruktur erstellen, die mögliche Qualitätskriterien in den Ergebnissen vorhält, aber die Ergebnisse noch nicht nach diesen Kriterien wertet.

Die allgemeingültigen Qualitätskriterien für Ergebnisse aus einer Informationsrecherche im Internet nach Georeferenzen sind in unserem Fall das Vertrauen zur Informationsquelle (Korrektheit der Information) und die Genauigkeit der Information. Letztere  unterteilt sich bezogen auf Georeferenzen in die Genauigkeit einer Ortsangabe (``Wo?'') und die Genauigkeit einer zeitlichen Angabe (``Wann?'').
\section{Probleme}
Die effiziente Geolokalisierung von Akteuren muss mit Hilfe von Computerprogrammen erfolgen. Sofern für diese Aufgabe keine besondere spezialisierte Informationsquelle zur Verfügung steht, werden hierfür im Allgemeinen Informationen aus dem Internet verwendet. Abhängig vom Zutrauen zur Korrektheit der Informationsquelle und dem Extraktionsverfahren mit dem die Informationen ermittelt werden, ändert sich die Qualität des Ergebnisses. Für eine Problemlose weiterverwendung der Daten aus einem Geolokalisierungsprozesses ist es unabdingbar, dass die Qualität der Ergebnisse bekannt ist.

Die Bewertung der Güte einer Informationsquelle ist zumindest teilweise eine subjektive Entscheidung, die sich im besten Fall auf möglichst viele Indizien stützt. Da unser Datenmodell außerdem für mehrere Implementierungen einer Geolokalisierung übergreifend gültig sein soll, ist es nicht sinnvoll die Güte einer Informationsquelle in einer zu granularen Einheit abzubilden. Gibt man die Güte einer Quelle zum Beispiel mit einer Fließkommazahl zwischen 0 (nicht vertrauenswürdig) bis 1 (voll vertrauenswürdig) an, können die Indikatoren von verschiedenen Quellen sehr nahe beieinander liegen (z.B. 0.8 und 0.83). Auf Grund der subjektiven Anteile in der Bewertung und unterschiedlicher Implementierung der Informationsextraktion in unterschiedlichen Programmen sagt ein so geringer Unterschied nichts darüber aus, welche Quelle eine bessere Güte hat. Wir haben uns daher stattdessen dazu entschieden, die Güte einer Quelle in relativ groben Einheiten zu unterscheiden, wie zum Beispiel ``sehr vertrauenswürdig'', ``vertrauenswürdig'', ``nicht vertrauenswürdig''. Da eine solch grobe Einteilung ein Rechnen mit den Güteindikatoren verhindert und die Abstände zwischen den Einheiten unbestimmt groß sein können, bieten sich hierfür Liekert-Skalen an. Eine genaue Festlegung findet sich im Abschnitt \ref{datenmodell} ``Datenmodell''.

\section{Datenmodell}
\label{datenmodell}

Die Datenstruktur kann in einem einfach gehaltenen Entity-Relationship-Modell dargestellt werden. Im Wesentlichen besteht die Basis-Entität aus grundlegenden Container-Elementen, die an essentiellen Stellen mit Informationen angereichert wurden. Die Datenstruktur wird in diesem Kapitel kurz erläutert.

Entity Relation Model

Den übergeordneten Datencontainer bildet die Reference-Klasse, sie hält zwei Attribute, OSM::Location und eine Liste von Discoveries vor. Eine Georeferenz (also eine Instanz der hier beschriebenen Reference-Klasse) liefert einen Verweis auf einen geografischen, real existierenden Ort auf der Welt und eine Liste von Hinweisen, wie und wo diese Referenz durch etwaige Suchalgorithmen gefunden und in den entsprechend gefundenen Dokumenten (beispielsweise auf Webseiten oder in semantischen Quellen wie dbpedia "referenziert" wurde.

Die OSM::Location-Klasse stellt eine sehr starke Abstraktion einer OpenStreetMaps Node\cite{OSMnode} dar. Im OpenStreetMaps-Datenmodell repräsentieren Nodes einen distinkten Punkt, der innerhalb des Systems einzigartig ist und mit Positionsdatenversehen ist. In unserem Datenmodell sind die konkreten Geokoordinaten nicht von essentieller Wichtigkeit (weil der Fokus auf Flächen, Regionen, Städten liegt, wichtig ist uns die Bennenbarkeit), daher persistieren wir in der OSM::Location-Klasse lediglich die node\_id, einen menschenlesbaren Namen und eine Repräsentation des Administration-Level-Konzepts von OpenStreetMaps\cite{OSMadminlevel}. Mit dem Administration-Level kann bestimmt werden, ob es sich bei der angegebenen Georeferenz etwa um einen Kontinent, einen Staat, ein Bundesland oder eine Stadt handelt.

Weiterhin enthält die Reference-Klasse wie erwähnt eine Liste von Discovery-Instanzen. In einer Discovery-Instanz wird festgehalten, wo (Source-Klasse) und in welchem zeitlichen Rahmen (Time-Klasse) diese Georeferenz an dem durch die OSM::Location-Klasse konkret festgelegten Ort referenziert wird. Weiterhin gibt es eine Discovery::Locator-Klasse, die die Rückverfolgbarkeit der Quelle sicherstellt. Die einzelnen Discovery-Komponenten werden im Folgenden noch genauer geschrieben.

Die Discovery::Source-Klasse ist ein Container für zwei Attribute: Zum einen hält ein identifier eine Referenz auf eine konkrete Datenquelle vor, beispielsweise dbp für dbPedia. Zum anderen hält der trustworthiness-Wert eine Einschätzung für die Zuverlässigkeit der Quelle vor. Beide Attribute ergeben sich aus dem Anwendungsfall und müssen pro Durchgang konfigurierbar gehalten sein. Der trustworthiness-Wert enthält Werte einer Likert-Skala (von "Sehr vertrauenswürdig" bis "Nahezu nicht vertrauenswürdig"), intern können die Werte dieser Skala auf Ganzzahlen (oder Enumeration-Values) abgebildet werden, allerdings verbietet sich das Rechnen auf Likert-Zahlen.

Die Discovery::Time-Klasse ist ein Container für die beiden Attribute representation und estimated\_precision. Der erste Wert gibt eine konkrete, wörtliche Fundstelle einer Zeitangabe wieder (also beispielsweise "1990ies", "1997" oder "1990-1996"). Der zweite Wert gibt auf einer Likert-Skala (von "sehr präzise" bis "sehr unpräzise") eine Einschätzung über die Genauigkeit der Zeitangabe - in Hinblick auf den Anwendungsfall - an. So wäre beispielsweise im Rahmen einer Epochenbeschreibung eine Jahresangabe "sehr präzise" und eine Angabe eines Jahrtausends "sehr unpräzise". Selbige Einschränkungen wie oben beschrieben in Hinblick auf die Likert-Skalen gelten auch hier. Die Einschätzung der Präzision muss ein Algorithmus liefern.

Der Discovery::Locator ist eine Klasse mit Elementen, die auf den genauen Fundort einer Georeferenz in einem Web-Dokument (oder ähnlichem) verweist. Ziel hierbei ist, im späteren Verlauf den Ursprung der Fundstelle maschinell oder durch den Menschen nachverfolgbar (oder belegbar, beispielsweise in einem journalistischen Anwendungsszenario) zu machen. Konkret beinhaltet die Discovery::Locator-Klasse die Felder url um mit dort eine Resource im Internet unabhängig von einem konkreten Schema referenzieren zu können, sowie ein byte\_offset, um die konkrete Stelle im Dokument, das durch die URL referenziert ist, anzuzeigen, als auch einen timestamp, um den Zeitpunkt des Fundes festzuhalten.
  
Es existiert weiterhin ein optionales Feld proof, das die Fund-Webresource base64-enkodiert in Kopie vorhält.

Alle oben genannten Klassen enthalten beliebig weitere Attribute, die vom Namen her unterschiedlich zu den bereits vorgestellten Attributen sein müssen. Hier hat der Entwickler die Möglichkeit anwendungsspezifische Daten in die Datenstruktur einzubetten um beispielsweise Geokodierungsschritte oder Umwandlungen einzusparen oder zusätzliche Meta-Informationen mitzuliefern.

% TODO Add citations for OSM

\section{Technologien}
\label{technologies}
Es liegt nahe, für die beschriebene Datenstruktur Schnittstellen bereitzustellen und bestehende Technologien zu verwenden, die den Gebrauch des Ganzen möglichst effizient und komfortabel gestalten und unterstützen. Hier soll kurz beschrieben werden, welche technischen Herangehensweisen in der Verwendung der genannten Datenstruktur sinnvoll erscheinen.

Es ist häufig von geografischen Referenzen die Rede, daher ist es sinnvoll, bestehende Geodatenbanken wie OpenStreetMaps zu verwenden, um Geo-Locations referenzieren zu können. Die Datenstruktur bedient sich, wie oben bereits erwähnt, einiger aus OpenStreetMaps entlehnter Konzepte, die direkt auf das sehr umfangreiche und freie Kartenprojekt abgebildet werden können. OpenStreetMaps selbst basiert zum überwiegenden Teil auf XML-Formaten, was die Verzahnung mit weiteren Webprojekten noch weiter vereinfachen kann.

Die Referenzimplementierung der Datenstruktur liegt in Ruby vor. Ruby ist eine Programmiersprache, die neben ihrem Duck-Typing-Konzept auch für leistungsfähige Zeichenkettenverarbeitungverarbeitung bekannt ist. Beide Konzepte werden dann nützlich, wenn unterschiedlichste textbasierte Datenquellen eingelesen, verarbeitet und zu einer Datenkomponente zusammengefasst werden sollen. Die Offenheit der Implementierung kann vorteilhaft sein, wenn Attribute oder standardisierte Methoden nachgefügt werden sollen. Dies ist beispielsweise auch dann der Fall, wenn zukünftig \textit{Reference}-Instanzen über Anwendungsprogrammierungsschnittstellen (\textit{API}s) weitergegeben werden sollen.

Als einzige wichtige standardisierte Methode auf dem Datentyp ist derzeit die \texttt{to\_json}-Methode, die die Felder aller Strukturen und Unterstrukturen in Javascript Object Notation\cite{jsonspec} zurück gibt. Dies ist sehr sinnvoll, für den Fall, dass Instanzen dieser Datenstrukturen typischerweise über \textit{Web-API}s geliefert werden sollen. Für die Heuristiken - als theoretisches Konstrukt, die diese Datenstrukturen bereitstellen - ist \texttt{to\_json} aber gewiss ohne Relevanz, findet sich aber beispielsweise in der Referenzimplementierung.



\bibliographystyle{asmems4}
\bibliography{paper}
\end{document}
