\documentclass[twocolumn,10pt]{asme2ej}
\usepackage{epsfig} %% for loading postscript figures
\usepackage[utf8]{inputenc}
\title{Ein Datenmodell für die Geolokalisierung von Akteuren} 
%%% first author
\author{Hendrik Bergunde
    \affiliation{
	Masterstudent, Freie Universität Berlin\\
    Email: hendrik.bergunde@fu-berlin.de
    }	
}
%%% second author
\author{Frank Zechert
    \affiliation{
	Masterstudent, Freie Universität Berlin\\
    Email: frank.zechert@fu-berlin.de
    }	
}
\begin{document}

\maketitle    

\begin{abstract}
{\it Akteure, zum Beispiel Firmen oder Personen, sind im Zeitraum ihres Lebens oder ihrer Existenz an unterschiedlichen Orten aktiv. Möchte man diese Akteure mit Hilfe von verschiedenen Techniken und Programmen lokalisieren, benötigt man ein einheitliches Datenmodell, in dem die Ergebnisse repräsentiert werden. Nur dann ist es möglich die verschiedenen Ergebnisse zu vereinigen und gemeinsam auszuwerten. In diesem Paper stellen wir ein Datenmodell für diesen Zweck vor, welches auf der einen Seite allgemein genug ist, um unabhängig von der verwendeten Technik zu sein, und auf der anderen Seite die Ergebnisse trotzdem mit wichtigen Qualitätsmerkmalen ausstattet.}
\end{abstract}

\section{Motivation und Ziele}
Motivation \& Ziele
  Georeferenzen aus verschiedenen Systemumgebungen abbildbar, vergleichbar, zusammenführbar machen
  Dazu Vertrauen und Genauigkeit (Qualitätskriterien?) ausdrücken
  
\section{Probleme}
"Probleme"
Warum zB Likertskalen? 
Rückverfolgbarkeit der Treffer (Quellenangabe?)

\section{Assoziierte Technologien}
"Assoziierte" Technologien 
OSM (-> schnell bei XML) % http://wiki.openstreetmap.org/wiki/Node
JSON

\section{Datenmodell}

Entity Relation Model

  Beschreibung der einzelnen Entitäten
    Referenz (nur "Rahmen")
    OSM-Location (node\_id und OSM Layer Verlinkung)
    Discovery (Rahmen für S und T)
    Discovery::Locator (URL, byteindex)
        "url": http://wikipedia.org/JebBush
        "byte\_offset": 12345 % optional
        "timestamp": ...
        "proof": { payload\_base64: "...", mimetype: "text/html" }
    Discovery::Source
    Discovery::Time

    Begründung Existenz anwendungsspezifische Felder überall (optionals) als K/V in das jeweilige Objekt reingehängt

  Festlegen der Skalen (TW)



\bibliographystyle{asmems4}
\bibliography{paper}
\end{document}
