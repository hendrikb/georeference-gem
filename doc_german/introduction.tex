\section{Einführung}
\label{introduction}
Es existiert ein Anwendungsfall, in dem ein Algorithmus verschiedenste Quellen im Internet nach benannten Orten (real existierende geografische Entitäten: Städte, Länder, Kontinente, etc.) durchsuchen soll. Diese Orte werden anschließend zueinander in Verbindung gebracht.

Ziel ist es natürlich, möglichst viele Quellen anbinden zu können. So sollen beispielsweise Texte in der Wikipedia oder im freien Internet (z.B. ausgehend von Suchmaschinen-Ergebnisseiten) nach benannten Orten durchsucht werden. Die Voraussetzung dafür ist aber, dass dieser \textit{Scraping}-Algorithmus verschiedenste Quellen anbinden kann.

Um diese Fundstellen aus unterschiedlichen Quellen vorhalten zu können, muss es eine Datenstruktur geben, die gewährleistet, dass - egal aus welche Quelle eine dieser Georeferenzen stammt - die Ergebnisse gleichartig gespeichert, weitergegeben und insbesondere miteinander verglichen werden können.

Diese Heuristik soll hier von verschiedensten Blickwinkeln aus betrachtet und erläutert werden.
