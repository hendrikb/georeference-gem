\section{Motivation und Ziele}

Die Geolokalisierung von Akteuren mit Hilfe von Daten aus dem Internet kann auf verschiedenste Weise maschinell oder händisch durchgeführt werden. Ziel dabei ist immer die Extraktion von Informationen, die auf den Aufenthaltsort des Akteurs zu einem gewissen Zeipunkt Hinweise geben könnten. Wenn verschiedene Systeme und Verfahren mit diesem Extraktionsprozess betraut werden, benötigt man ein einheitliches Datenformat für die im Ergebnis gefundenen Georeferenzen. Möchte man auf einer möglichst vollständigen Datenmenge an Georeferenzen arbeiten, muss man die Ergebnisse von verschiedenen Ansätzen zusammenführen und miteinander vergleichen können.

Wie genau verschiedene Ergebnisse miteinander verglichen oder zusammengeführt werden, muss hierbei jedoch immer vom Anwendungsfall abhängen. Je nach Anwendungsfall kann sich das Interesse an unterschiedlichen Qualitätskriterien der Ergebnisse stark unterscheiden. Wir wollen daher eine Datenstruktur erstellen, die mögliche Qualitätskriterien in den Ergebnissen vorhält, aber die Ergebnisse noch nicht nach diesen Kriterien wertet.

Die allgemeingültigen Qualitätskriterien für Ergebnisse aus einer Informationsrecherche im Internet nach Georeferenzen werden in der hier beschriebenen Herangehenseise anhand der im Folgenden beschriebenen Qualitätsdimensionen definiert: Zum einen muss ein Maß an Vertrauen zur Informationsquelle (Korrektheit der Information) angegeben werden können. Zum anderen muss die Genauigkeit der Information der eigentlichen Referenz bewertet werden können. Diese Angabe unterteilt sich, bezogen auf Georeferenzen in weitgehend unstrukturierten Webdokumenten, in die Genauigkeit einer Ortsangabe (``Wo?'') und die Genauigkeit einer zeitlichen Angabe (``Wann?'').
