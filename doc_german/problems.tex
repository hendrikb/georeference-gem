\section{Probleme}
Die effiziente Geolokalisierung von Akteuren muss mit Hilfe von Computerprogrammen erfolgen. Sofern für diese Aufgabe keine besondere spezialisierte Informationsquelle zur Verfügung steht, werden hierfür im Allgemeinen Informationen aus dem Internet verwendet. Abhängig vom Zutrauen zur Korrektheit der Informationsquelle und dem Extraktionsverfahren mit dem die Informationen ermittelt werden, ändert sich die Qualität des Ergebnisses. Für eine Problemlose weiterverwendung der Daten aus einem Geolokalisierungsprozesses ist es unabdingbar, dass die Qualität der Ergebnisse bekannt ist.

Die Bewertung der Güte einer Informationsquelle ist zumindest teilweise eine subjektive Entscheidung, die sich im besten Fall auf möglichst viele Indizien stützt. Da unser Datenmodell außerdem für mehrere Implementierungen einer Geolokalisierung übergreifend gültig sein soll, ist es nicht sinnvoll die Güte einer Informationsquelle in einer zu granularen Einheit abzubilden. Gibt man die Güte einer Quelle zum Beispiel mit einer Fließkommazahl zwischen 0 (nicht vertrauenswürdig) bis 1 (voll vertrauenswürdig) an, können die Indikatoren von verschiedenen Quellen sehr nahe beieinander liegen (z.B. 0.8 und 0.83). Auf Grund der subjektiven Anteile in der Bewertung und unterschiedlicher Implementierung der Informationsextraktion in unterschiedlichen Programmen sagt ein so geringer Unterschied nichts darüber aus, welche Quelle eine bessere Güte hat. Wir haben uns daher stattdessen dazu entschieden, die Güte einer Quelle in relativ groben Einheiten zu unterscheiden, wie zum Beispiel ``sehr vertrauenswürdig'', ``vertrauenswürdig'', ``nicht vertrauenswürdig''. Da eine solch grobe Einteilung ein Rechnen mit den Güteindikatoren verhindert und die Abstände zwischen den Einheiten unbestimmt groß sein können, bieten sich hierfür Liekert-Skalen an. Eine genaue Festlegung findet sich im Abschnitt \ref{datenmodell} ``Datenmodell''.

Neben der Korrektheit einer Information aus einer Quelle spielt auch der Zeitpunkt eine Rolle, auf den sich eine Information aus einer Quelle bezieht. Akteure, deren aufenthaltsorte man mittels Geolokalisierung finden möchte, bewegen sich im Laufe der Zeit zwischen verschiedenen Orten. Die Information, dass sich ein Akteur in New York (U.S.A.) befindet, ist erst aussagekräftig, wenn wir wissen, wann dies der Fall war. Im günstigsten Fall ist eine solche Information mit einem konkreten Datum (Zeitpunkt) oder mit eindeutigen Anfangs- und Enddaten (Zeitintervall) annotiert. Ist dies nicht der Fall, kann die Präzision der Information stark nachlassen. Ohne jegliche Angabe eines Zeitpunktes für einen Auffenthaltsort einer Person, lässt sich nur die Aussage treffen, das eine Person zwischen ihrem Geburts- und Todesdatum zu einem beliebigen Zeitpunkt an einem Ort war. Ähnlich zu den Liekerst-Skalen für die Qualität der Informationsquelle ist es auch hier sinnvoll, eine etwas gröbere Abstufung für die Genauigkeit des Zeitpunktes anzugeben. In diesem Fall beginnend bei der genausten Angabe, nämlich ein konkretes ``Datum'', über etwas ungenauere Angaben wie ``Jahr`` oder ``Jahrzehnt`` bis hin zur ungenauesten Angabe ``Jahrhundert``. Daten die noch ungenauer sind und daher nicht einem Jahrhundert zugeordnet werden können sind ohne Bedeutung und werden hier nicht weiter beachtet.

